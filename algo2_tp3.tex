\documentclass[a4paper,10pt]{article}
\usepackage[paper=a4paper, hmargin=1.5cm, bottom=1.5cm, top=3.5cm]{geometry}
\usepackage[latin1]{inputenc}
\usepackage[T1]{fontenc}
\usepackage[spanish]{babel}
\usepackage{xspace}
\usepackage{xargs}
\usepackage{ifthen}
\usepackage{algorithm}% http://ctan.org/pkg/algorithms
\usepackage{algpseudocode}% http://ctan.org/pkg/algorithmicx
\usepackage{verbatim}
\usepackage{listings}
\usepackage{aed2-tad,aed2-symb,aed2-itef,caratula}

% lo nuestro

\lstdefinestyle{alg}{tabsize=4, frame=single, escapeinside=\'\', framesep=10pt}

\newcommand{\alg}[3]{\hangindent=\parindent#1 (#2) \ifx#3\empty\else$\rightarrow$ res: #3\fi}
\newcommand\ote[1]{\hspace*{\fill}~\mbox{O(#1)}\penalty -9999 }
\newcommand\ofi[1]{\ensuremath{\textbf{Complejidad}: #1}}

% lo otro

\newcommand{\moduloNombre}[1]{\textbf{#1}}

\let\NombreFuncion=\textsc
\let\TipoVariable=\texttt
\let\ModificadorArgumento=\textbf
\newcommand{\res}{$res$\xspace}
\newcommand{\tab}{\hspace*{7mm}}

\newcommandx{\TipoFuncion}[3]{%
  \NombreFuncion{#1}(#2) \ifx#3\empty\else $\to$ \res\,: \TipoVariable{#3}\fi%
}
\newcommand{\In}[2]{\ModificadorArgumento{in} \ensuremath{#1}\,: \TipoVariable{#2}\xspace}
\newcommand{\Out}[2]{\ModificadorArgumento{out} \ensuremath{#1}\,: \TipoVariable{#2}\xspace}
\newcommand{\Inout}[2]{\ModificadorArgumento{in/out} \ensuremath{#1}\,: \TipoVariable{#2}\xspace}
\newcommand{\Aplicar}[2]{\NombreFuncion{#1}(#2)}

\newlength{\IntFuncionLengthA}
\newlength{\IntFuncionLengthB}
\newlength{\IntFuncionLengthC}
%InterfazFuncion(nombre, argumentos, valor retorno, precondicion, postcondicion, complejidad, descripcion, aliasing)
\newcommandx{\InterfazFuncion}[9][4=true,6,7,8,9]{%
  \hangindent=\parindent
  \TipoFuncion{#1}{#2}{#3}\\%
  \textbf{Pre} $\equiv$ \{#4\}\\%
  \textbf{Post} $\equiv$ \{#5\}%
  \ifx#6\empty\else\\\textbf{Complejidad:} #6\fi%
  \ifx#7\empty\else\\\textbf{Descripci�n:} #7\fi%
  \ifx#8\empty\else\\\textbf{Aliasing:} #8\fi%
  \ifx#9\empty\else\\\textbf{Requiere:} #9\fi%
}

\newenvironment{Interfaz}{%
  \parskip=2ex%
  \noindent\textbf{\Large Interfaz}%
  \par%
}{}

\newenvironment{Representacion}{%
  \vspace*{2ex}%
  \noindent\textbf{\Large Representaci�n}%
  \vspace*{2ex}%
}{}

\newenvironment{Algoritmos}{%
  \vspace*{2ex}%
  \noindent\textbf{\Large Algoritmos}%
  \vspace*{2ex}%
}{}


\newcommand{\Titulo}[1]{
  \vspace*{1ex}\par\noindent\textbf{\large #1}\par
}

\newenvironmentx{Estructura}[2][2={estr}]{%
  \par\vspace*{2ex}%
  \TipoVariable{#1} \textbf{se representa con} \TipoVariable{#2}%
  \par\vspace*{1ex}%
}{%
  \par\vspace*{2ex}%
}%

\newboolean{EstructuraHayItems}
\newlength{\lenTupla}
\newenvironmentx{Tupla}[1][1={estr}]{%
    \settowidth{\lenTupla}{\hspace*{3mm}donde \TipoVariable{#1} es \TipoVariable{tupla}$($}%
    \addtolength{\lenTupla}{\parindent}%
    \hspace*{3mm}donde \TipoVariable{#1} es \TipoVariable{tupla}$($%
    \begin{minipage}[t]{\linewidth-\lenTupla}%
    \setboolean{EstructuraHayItems}{false}%
}{%
    $)$%
    \end{minipage}
}

\newcommandx{\tupItem}[3][1={\ }]{%
    %\hspace*{3mm}%
    \ifthenelse{\boolean{EstructuraHayItems}}{%
        ,#1%
    }{}%
    \emph{#2}: \TipoVariable{#3}%
    \setboolean{EstructuraHayItems}{true}%
}

\newcommandx{\RepFc}[3][1={estr},2={e}]{%
  \tadOperacion{Rep}{#1}{bool}{}%
  \tadAxioma{Rep($#2$)}{#3}%
}%

\newcommandx{\Rep}[3][1={estr},2={e}]{%
  \tadOperacion{Rep}{#1}{bool}{}%
  \tadAxioma{Rep($#2$)}{true \ssi #3}%
}%

\newcommandx{\Abs}[5][1={estr},3={e}]{%
  \tadOperacion{Abs}{#1/#3}{#2}{Rep($#3$)}%
  \settominwidth{\hangindent}{Abs($#3$) \igobs #4: #2 $\mid$ }%
  \addtolength{\hangindent}{\parindent}%
  Abs($#3$) \igobs #4: #2 $\mid$ #5%
}%

\newcommandx{\AbsFc}[4][1={estr},3={e}]{%
  \tadOperacion{Abs}{#1/#3}{#2}{Rep($#3$)}%
  \tadAxioma{Abs($#3$)}{#4}%
}%


\newcommand{\DRef}{\ensuremath{\rightarrow}}



\begin{document}



% Estos comandos deben ir antes del \maketitle
\materia{Algoritmos y Estructuras de Datos II} % obligatorio
\submateria{Segundo Cuatrimestre de 2014} % opcional
\titulo{Trabajo Pr�ctico 2} % obligatorio
\subtitulo{Dise\~no} % opcional
\grupo{Grupo 17} % opcional 

\integrante{Alejandro Candioti}{784/13}{amcandio@gmail.com} % obligatorio 
\integrante{Guido Tamborindeguy}{584/13}{guido@tamborindeguy.com.ar } % obligatorio 
\integrante{Martin Jedwabny}{885/13}{martinj@live.com.ar} % obligatorio 
\integrante{Lucas Puterman}{830/13}{lucasputerman@gmail.com} % obligatorio 

\maketitle

\pagebreak
\tableofcontents
\pagebreak

%%%%%%%%%%%%%%%%%%%%%%%%%%

\section{M�dulo Mapa}

%%%%%%%%%%%%%%%%%%%%%%%%%%

\subsection{Interfaz}

  \textbf{se explica con}: \tadNombre{Mapa}.

  \textbf{g�neros}: \TipoVariable{map}.

  \subsubsection{Operaciones b�sicas de mapa}

  \InterfazFuncion{Vacio}{}{map}
  [true]
  {$res$ $\igobs$ vacio()}
  [$O(1)$]
  [crea un mapa nuevo]
  []
  
  ~  

  \InterfazFuncion{Agregar}{\In{e}{string}, \Inout{m}{map}}{}
  [$e \not \in$ estaciones($m$) $\land$ |$e$| > $ $ 0 $\land$ $m \igobs m_0$]
  {$m \igobs$ agregar($e$, $m_0$)}
  [$O(|e|)$]
  [agrega una estacion al mapa]
  [] %TODO: (martin)
  
  ~  

  \InterfazFuncion{Conectar}{\In{e1}{string}, \In{e2}{string}, \In{r}{restriccion}, \Inout{m}{map}}{}
  [$e1 \ne e2 \land$ ($e1 \in$ estaciones($m$) $\land$ $e2 \in$ estaciones($m$)) $\yluego$ ($\neg$ conectadas?($e1$, $e2$)) $\land$ $m \igobs m_0$]
  {$m \igobs$ conectar($e1$, $e2$, $r$, $m_0$)}
  [$O(|e1| + |e2|)$]
  [conecta dos estaciones previamente agregadas con su respectiva restriccion para la senda que forman]
  []
  
  ~  

  \InterfazFuncion{Esta?}{\In{e}{string}, \In{m}{map}}{bool}
  [true]
  {$res \igobs e \in$ estaciones($m$)}
  [$O(|e|)$]
  [verifica si la estacion fue agregada a la ciudad]
  []
  
  ~  

  \InterfazFuncion{Conectadas?}{\In{e1}{string}, \In{e2}{string}, \In{m}{map}}{bool}
  [($e1 \in$ estaciones($m$) $\land$ $e2 \in$ estaciones($m$)) $\land$ $m \igobs m_0$]
  {$res \igobs$ conectadas?($e1$, $e2$, $m$)}
  [$O(|e1|+|e2|)$]
  [Se fija si las dos estaciones estan conectadas segun el mapa]
  []
  
  ~  

  \InterfazFuncion{Restriccion}{\In{e1}{string}, \In{e2}{string}, \In{m}{map}}{restriccion}
  [($e1 \in$ estaciones($m$) $\land$ $e2 \in$ estaciones($m$)) $\yluego$ (conectadas?($e1$, $e2$))]
  {$res \igobs$ restriccion($e1$, $e2$, $m$)}
  [$O(|e1| + |e2|)$]
  [devuelve la restriccion correspondiente a la senda que conecta las dos estaciones en el mapa]
  []
  
  ~  

  \InterfazFuncion{Estaciones}{\In{m}{map}}{itLista(string)}
  [true]
  {SecuSuby(res) $\igobs$ estaciones(m)}
  [$O(1)$]
  [devuelve un iterador para las estaciones del mapa]
  []
  
  ~  

  \InterfazFuncion{Sendas}{\In{m}{map}}{itLista(tupla(string, string))}
  [true]
  {esPermutacion?(res, sendas(m))}
  [$O(1)$]
  [devuelve un iterador para los pares de estaciones que forman una senda]
  []
  
  ~  

  \subsubsection{Operaciones Auxiliares del TAD}

  \tadOperacion{sendas}{mapa/m}{lista(tupla(string, string))}{}
  \tadAxioma{sendas($m$)}{
    compararTodos(estaciones($m$), estaciones($m$))
  }

  ~     
  
  \tadOperacion{compararTodos}{lista(estacion)/l1, lista(estacion)/l2, mapa/m}{lista(tupla(string, string))}{estan?(l1, estaciones(m)) $\land$ estan?(l2, estaciones(m))}
  \tadAxioma{compararTodos($l1, l2, m$)}{
    \IF vacia?($l1$) THEN <> ELSE compararUno(prim($l1$), $l2$) \& compararTodos(fin($l1$, $l2$)) FI
  }
  
  ~     
  
  \tadOperacion{compararUno}{estacion/e, lista(estacion)/l, mapa/m}{lista(tupla(string, string))}{esta?(e, estaciones(m)) $\land$ estan?(l, estaciones(m))}
  \tadAxioma{compararUno($e, l, m$)}{
    \IF vacia?($l$) THEN 
      <> 
    ELSE
      {
      \IF $e1$ > prim($l$) $\land$ conectadas?($e1, prim($l$), m$) THEN
        <$e1, prim($l$)$> $ $ \textbullet $ $ compararUno($e$, fin($l$))
      ELSE
        compararUno($e$, fin($l$))
      FI
      }
    FI
  }
  
  ~     
  
  \tadOperacion{estan?}{lista($\alpha$)/l1, lista($\alpha$)/l2}{bool}{}
  \tadAxioma{estan?($l1, l2$)}{
    vacia?($l2$) $\oluego$ ($\neg$vacia?($l1$) $\yluego$ (esta?(prim($l1$), $l2$) $\land$ estan?(fin($l1$), sacar(prim($l1$), $l2$))))
  }
  
  ~     
  
  \tadOperacion{sacar}{$\alpha$/e, lista($\alpha$)/l}{bool}{}
  \tadAxioma{sacar($e, l$)}{
    \IF vacia?($l$) THEN <> ELSE {
      \IF prim($l$ = $e$) THEN
        sacar($e$, fin($l$))
      ELSE
        $e$ \textbullet $ $ sacar($e$, fin($l$))
      FI
    } FI
  }
  Obs: $\alpha$ debe ser comparable con la funcion =.
  
  ~     
  
  \tadOperacion{esPermutacion?}{lista($\alpha$)/l1, lista($\alpha$)/l2}{bool}{}
  \tadAxioma{esPermutacion?($l1, l2$)}{
    estan?($l1, l2$) $\land$ estan?($l2, l1$)
  }

%%%%%%%%%%%%%%%%%%%%%%%%%%

\subsection{Representacion}
  
  \subsubsection{Representaci�n de mapa}

  \begin{Estructura}{map}[estr]
    \begin{Tupla}[estr]
      \tupItem{estaciones}{lista(string)}%
      \tupItem{\\ sendas}{lista(tupla(e1: string, e2: string))}%
      \tupItem{\\ restricciones}{dicc$_{T}$(dicc$_{T}$(restriccion))}%
    \end{Tupla}
  \end{Estructura}

  \subsubsection{Invariante de Representaci\'on}
  
  \renewcommand{\labelenumi}{(\Roman{enumi})}
  
  \begin{enumerate}
  \item Las estaciones son las mismas que las claves de las sendas
  \item No esta definida una clave del diccionario dentro de sus diccionarios hijos
  \item Los diccionarios hijos de una clave estan definidos en el diccionario original
  \item Las claves de los hijos diccionarios de un item del diccionario tienen menor orden lexicografico que el padre
  \item Las combinaciones definidas en las restricciones son las mismas que la lista de sendas
  \end{enumerate}

  \Rep[estr][e]{
    ($\forall a$: string)($a \in e$.estaciones $<=>$ def?($a$, $e$.restricciones) ) $\land$ \\ 
    ($\forall c$, $s$: string)(def?($c$, $e$.restricciones) $\impluego$ \\ 
    ($\neg$def?($c$, obtener($c$, $e$.restricciones)) $\land$ \\ 
    (def?($s$, obtener($c$, $e$.restricciones)) $\impluego$ (def?($s$, $e$) $\land$ $s < c$)))) $\land$ \\ 
    ($\forall c$, $s$: string)(def?($c$, $e$.restricciones) $\impluego$ \\ 
    (def?($s$, obtener($c$, $e$.restricciones)) <=> <$c$, $s$> $\in$ $e$.sendas))}\mbox{}

  \subsubsection{Funci\'on de Abstracci\'on}
 
  \Abs[estr]{mapa}[e]{m}{
    m.estaciones $\igobs$ e.estaciones $\land$ \\ 
    ($\forall c$, $s$: string)(($c$ $\in$ estaciones($e$) $\land$ $s$ $\in$ estaciones($e$) $\land$ $c < s$) $\Rightarrow$ \\ 
                              ((def?($s$, obtener($c$, $e$)) $\igobs$ conectadas?($c$, $s$, $m$)) $\yluego$ \\ 
                              (def?($s$, obtener($c$, $e$)) $\impluego$ (obtener($s$, obtener($c$, $e$)) $\igobs$ restriccion($c$, $s$, $m$)))))
  }

%%%%%%%%%%%%%%%%%%%%%%%%%%

\subsection{Algoritmos}

\lstset{style=alg}

%{Vacio}{}{map}
\begin{lstlisting}[mathescape]
'\alg{iVacio}{}{estr}'

    res.estaciones $\leftarrow$ Vacia() '\ote{1}'
    res.sendas $\leftarrow$ CrearDicc() '\ote{1}'

'\ofi{O(1)}'
\end{lstlisting}

\begin{lstlisting}[mathescape]
'\alg{iAgregar}{\In{e}{string}, \Inout{m}{estr}}{}'

    Agregar(e, m.estaciones) '\ote{1}'
    Definir(e, m.sendas) '\ote{|e|}'

'\ofi{O(|e|)}'
\end{lstlisting}

\begin{lstlisting}[mathescape]
'\alg{iConectar}{\In{e1}{string}, \In{e2}{string}, \In{r}{restriccion}, \Inout{m}{estr}}{}'
  
  if e1 < e2 then '\ote{1}'
    Conectar(e2, e1, r, m) '\ote{|e1| + |e2|}'
  else
    if $\neg$Definido?(e1, m.restricciones) then '\ote{|e1|}'
      Definir(e1, CrearDicc(), m.restricciones) '\ote{|e1|}'
    end if
    Definir(e2, r, Obtener(e1, m.restricciones)) '\ote{|e1| + |e2|}'
    AgregarAdelante(m.sendas, <e1, e2>)
  end if

'\ofi{O(|e1| + |e2|)}'
\end{lstlisting}

\begin{lstlisting}[mathescape]
'\alg{iEsta?}{\In{e}{string}, \In{m}{map}}{bool}'
  
  res$\leftarrow$ Definido?(e, m.restricciones) '\ote{|e|}'

'\ofi{O(|e|)}'
\end{lstlisting}

\begin{lstlisting}[mathescape]
'\alg{iConectadas?}{\In{e1}{string}, \In{e2}{string}, \In{m}{map}}{bool}'
  
  res$\leftarrow$ Definido?(e1, m.restricciones) $\yluego$
        Definido?(e2, Obtener(e1, m.restricciones)) '\ote{|e1| + |e2|}'

'\ofi{O(|e1| + |e2|)}'
\end{lstlisting}

\begin{lstlisting}[mathescape]
'\alg{iRestriccion}{\In{e1}{string}, \In{e2}{string}, \In{m}{map}}{restriccion}'
  
  if e1 < e2 then '\ote{1}'
    Restriccion(e2, e1, m) '\ote{|e1| + |e2|}'
  else
    res$\leftarrow$ Obtener(e2, Obtener(e1, m.restricciones)) '\ote{|e1| + |e2|}'
  end if

'\ofi{O(|e1| + |e2|)}'
\end{lstlisting}

\begin{lstlisting}[mathescape]
'\alg{iEstaciones}{\In{m}{map}}{itLista(string)}'
  
  res$\leftarrow$ CrearIt(e.estaciones) '\ote{1}'

'\ofi{O(1)}'
\end{lstlisting}

\begin{lstlisting}[mathescape]
'\alg{iSendas}{\In{m}{map}}{itLista(tupla(string, string))}'
  
  res$\leftarrow$ CrearIt(e.sendas) '\ote{1}'

'\ofi{O(1)}'
\end{lstlisting}

%%%%%%%%%%%%%%%%%%%%%%%%%%


\pagebreak

\section{M�dulo DiccionarioString($\alpha$)}

%%%%%%%%%%%%%%%%%%%%%%%%%%

\subsection{Interfaz}
  
  \textbf{par\'{a}metros formales} \hangindent=2 \parindent \\
  \parbox{1.7cm}{\textbf{g�neros}} $\alpha$
  
  \textbf{se explica con}: \tadNombre{Diccionario(string, $\alpha$)}.
 
  \textbf{g�neros}: \TipoVariable{dicc$_T$$(\alpha)$}.

  \subsubsection{Operaciones b\'{a}sicas}
  
  \InterfazFuncion{CrearDicc}{}{dicc$_T$$(\alpha)$}%
  [true]
  {$res$ $\igobs$ vacio}%
  [$O(1)$]
  [crea un diccionario vacio]
  []

  ~

  \InterfazFuncion{Definido?}{\In{c}{string}, \In{d}{dicc$_T$$(\alpha)$})}{bool}
  [true]
	{$res$ $\igobs$ def?($c$, $d$)}
	[$O(|c|)$]
	[devuelve si la clave fue previamente definida en el diccionario]
	[]
  
  ~

  \InterfazFuncion{Definir}{\In{c}{string} , \In{s}{$\alpha$}, \Inout{d}{dicc$_T$$(\alpha)$}}{}
  [$ d \igobs d_0 $]
  {$ d \igobs$ definir($c, s, d_0$)}
  [$O$($|c| +$ copy($s$)) ]
  [define la clave $c$ con el significado $s$ en $d$]
  []
  
  ~

	\InterfazFuncion{Obtener}{\In{c}{string}, \In{d}{dicc$_T$($\alpha$)}}{$\alpha$}
	[def?($c$, $d$)]
	{alias($res$ $\igobs$ obtener($c$, $d$))}
	[$O(|c|)$]
	[devuelve el significado correspondiente a la clave en el diccinario]
	[$res$ es modificable si y solo si $d$ es modificable.]
  
  ~
	
  Obs: copy es una funci\'{o}n  de $\alpha$ en que devuelve el costo de copiar un elemento del g\'{e}nero $\alpha$.

%%%%%%%%%%%%%%%%%%%%%%%%%%

\subsection{Representaci\'on}

\subsubsection{Representaci\'on de DiccionarioString($\alpha$)}

  \begin{Estructura}{dicc$_T$$(\alpha)$}[estr donde \TipoVariable{estr} es \TipoVariable{puntero(nodo)}]
	\begin{Tupla}[nodo]
			 \tupItem{significado}{puntero$(\alpha)$}
			 \tupItem{\\ caracteres}{arreglo[256] de puntero(nodo)}
		\end{Tupla}
	\end{Estructura}

\subsubsection{Invariante de Representaci\'on}

\renewcommand{\labelenumi}{(\Roman{enumi})}

\begin{enumerate}
\item Todas las posiciones del arreglo de caracteres est\'{a}n definidas.
\item No hay claves de 0 caracteres. Esto es, el nodo raiz tiene el campo significado \TipoVariable{NULL}.
\item No hay ciclos en el trie. Esto es, existe un n\'{u}mero natural $n$ tal que la cantidad de niveles del \'{a}rbol est\'{a} acotada por $n$.
\item Dado un nodo cualquiera del trie, existe un \'{u}nico camino desde la ra\'{i}z hasta dicho nodo.
\end{enumerate}

\Rep[estr][e]{\\
  ($e$ $\rightarrow$ $significado$ $=$ NULL) $\wedge$\\
  ($\forall i$: nat)($i < 256$ $\implies$ definido?(e$\rightarrow$ caracteres, $i$)) $\yluego$ \\
  ($\exists n$: nat)(finaliza$(e, n)$) $\yluego$ \\
  ($\forall p$, $q$: puntero(nodo))($p$ $\in$ punteros$(e)$ $\wedge$ $q$ $\in$ (punteros$(e) - \{p\}$) $\implies p \neq q)$}\mbox{}

  ~     
  
\tadOperacion{finaliza}{estr/e, nat}{bool}{($\forall i$: nat) ($i < 256 \implies$ definido?($e\rightarrow caracteres, i$))}
	\tadAxioma{finaliza($e,n$)}{$n > 0 \yluego (e =$ NULL $\oluego$ finalizaAux($e\rightarrow caracteres, n - 1, 0))$ }

  ~
  
\tadOperacion{finalizaAux}{ad(puntero(nodo))/a, nat, nat/k}{bool}{$k \leq$ tam($a$)}
	\tadAxioma{finalizaAux($a, n, k$)}{\IF $k = $ tam($a$) THEN true ELSE finaliza($e\rightarrow caracteres[k], n$) $\wedge$ finalizaAux($a, n, k + 1$) FI}
  
  ~
	
\tadOperacion{punteros}{estr/e}{multiconj(puntero(nodo))}{}
	\tadAxioma{punteros($e$)}{\IF $e = $ NULL THEN $\emptyset$ ELSE punterosAux($e\rightarrow caracteres$, 0) FI}

  ~
	
\tadOperacion{punterosAux}{ad(puntero(nodo))/a , nat/k}{multiconj(puntero(nodo))}{$k \leq$ tam($a$)}
	\tadAxioma{punterosAux($a,k$)}{\IF $k = $ tam($a$) THEN $\emptyset$ ELSE ({\IF $a[k] =$ NULL THEN $\emptyset$ ELSE  Ag($a[k]$, punteros($a[k]$)) FI}) $\cup$ punterosAux($a, k + 1$) FI}

\subsubsection{Funci\'on de Abstracci\'on}

\Abs[estr]{dicc$_T$($\alpha$)}[e]{d}{($\forall c$: string)(def?($c, d$) $=$ esClave?($c, e$) $\yluego$ \\
 (def?($c, d$) \impluego obtener($c, d$) $=$ significado($c, e$)))}

 	~
 
 \tadOperacion{esClave?}{string/c, estr/e}{bool}{Rep($e$)}
 	\tadAxioma{esClave?($c, e$)}{\IF vac\'{i}a?($c$) THEN $e\rightarrow significado$ $\neq$ NULL ELSE  $e\rightarrow caracteres$[ord(prim($c$))] $\neq$ NULL \yluego esClave?(fin($c$), $e\rightarrow caracteres$[ord(prim($c$))]) FI}

	~

  \tadOperacion{significado}{string/c, estr/e}{$\alpha$}{Rep($e$) $\wedge$ esClave?($c, e$)}
  	\tadAxioma{significado($c,e$)}{\IF vac\'{i}a?($c$) THEN $*$($e\rightarrow significado$) ELSE significado(fin($c$), $e\rightarrow caracteres$[ord(prim($c$))]) FI}

%%%%%%%%%%%%%%%%%%%%%%%%%%

\subsection{Algoritmos}

\lstset{style=alg}

\begin{lstlisting}[mathescape]
'\alg{iCrearDicc}{}{estr}'

     (res$\rightarrow$ significado) $\leftarrow$ NULL '\ote{1}'
     (res$\rightarrow$ caracteres) $\leftarrow$ CrearArreglo(256) '\ote{1}'
     for i $\leftarrow$ 0 to 255 do '\ote{1}'
         (res$\rightarrow$ caracteres[i]) $\leftarrow$ NULL '\ote{1}'
     end for

'\ofi{O(1)}'
\end{lstlisting}

\begin{lstlisting}[mathescape]
'\alg{iDefinido?}{\In{c}{string}, \In{d}{estr}}{bool}'

     nat i $\leftarrow$ 0 '\ote{1}'
     bool esta $\leftarrow$ true '\ote{1}'
     puntero(nodo) actual $\leftarrow$ d '\ote{1}'
     while i < Longitud(c) $\yluego$ esta do '\ote{|c|}'
         if actual$\rightarrow$ caracteres[ord(c[i])] = NULL  then '\ote{1}'
             esta $\leftarrow$ false '\ote{1}'
         end if
         actual $\leftarrow$ (actual $\rightarrow$ caracteres[ord(c[i])]) '\ote{1}'
         i $\leftarrow$ i + 1 '\ote{1}'
     end while  
     res $\leftarrow$ (esta $\yluego$ $\neg$(actual$\rightarrow$ significado = NULL)) '\ote{1}'

'\ofi{O(|c|)}'
\end{lstlisting}

\begin{lstlisting}[mathescape]
'\alg{iDefinir}{\In{c}{string}, \In{s}{$\alpha$}, \Inout{d}{estr}}{}'

     nat i $\leftarrow$ 0 '\ote{1}'
     puntero(nodo) actual $\leftarrow$ d '\ote{1}'
     while i < Longitud(c) do '\ote{|c|}'
         if actual$\rightarrow$ caracteres[ord(c[i])] = NULL  then '\ote{1}'
             (actual$\rightarrow$ caracteres[ord(c[i])]) $\leftarrow$ CrearDicc() '\ote{1}'
         end if
         actual $\leftarrow$ (actual$\rightarrow$ caracteres[ord(c[i])]) '\ote{1}'
         i $\leftarrow$ i + 1 '\ote{1}'
     end while
     (actual$\rightarrow$ significado) $\leftarrow$ &Copiar(s) '\ote{copy(s)}'

'\ofi{O(|c|+copy(s))}'
\end{lstlisting}

\begin{lstlisting}[mathescape]
'\alg{iObtener}{\In{c}{string}, \In{d}{estr}}{\TipoVariable{$\alpha$}}'

     nat i $\leftarrow$ 0 '\ote{1}'
     puntero(nodo) actual $\leftarrow$ d '\ote{1}'
     while i < Longitud(c) do  '\ote{|c|}'
         actual $\leftarrow$ (actual$\rightarrow$ caracteres[ord(c[i])]) '\ote{1}'
         i $\leftarrow$ i + 1 '\ote{1}'
     end while 
     res $\leftarrow$ *(actual$\rightarrow$ significado) '\ote{1}'

'\ofi{O(|c|)}'
\end{lstlisting}

%%%%%%%%%%%%%%%%%%%%%%%%%%

\subsection{Servicios Usados}	

$\alpha$ debe proveer la operaci\'{o}n:
	
\InterfazFuncion{Copiar}{\In{s}{$\alpha$}}{$\alpha$}
{$res$ \igobs $s$}
[O(copy($s$))]\\
Donde se copia $s$, de modo que no haya aliasing entre $s$ y $res$

\pagebreak

%%%%%%%%%%%%%%%%%%%%%%%%%%

\section{M�dulo Conjunto de String}

%%%%%%%%%%%%%%%%%%%%%%%%%%

\subsection{Interfaz}

  \textbf{se explica con}: \tadNombre{conj(string)}.

  \textbf{g�neros}: \TipoVariable{conj$_T$}.

  \subsubsection{Operaciones b�sicas de conjunto de string}

  \InterfazFuncion{Vacio}{}{conj$_T$}
  [true]
  {$res$ $\igobs$ $\emptyset$}
  [$O(1)$]
  [crea un conjunto]
  []
  
  ~  

  \InterfazFuncion{Agregar}{\In{s}{string}, \Inout{c}{conj$_T$}}{}
  [true]
  {$c \igobs$ Ag($s, c$)}
  [$O(|s|)$]
  [agrega un string al conjunto]
  []
  
  ~  

  \InterfazFuncion{Borrar}{\In{s}{string}, \Inout{c}{conj$_T$}}{}
  [$s \in c $]
  {$\neg s \in c $}
  [$O(|s|)$]
  [borra un string del conjunto]
  []

  ~  

  \InterfazFuncion{Pertenece?}{\In{s}{string}, \In{c}{conj$_T$}}{bool}
  [true]
  {$res \igobs s \in c $}
  [$O(|s|)$]
  [verifica si el string pertenece al conjunto]
  []

  ~  

%%%%%%%%%%%%%%%%%%%%%%%%%%

\subsection{Representacion}
  
  \subsubsection{Representaci�n de conjunto de string}

  \begin{Estructura}{map}[estr]
    donde \TipoVariable{estr} es \TipoVariable{dicc$_T$(bool)}
  \end{Estructura}

  \subsubsection{Invariante de Representaci\'on}

  \Rep[estr][e]{true}\mbox{}

  \subsubsection{Funci\'on de Abstracci\'on}
 
  \Abs[estr]{conj(string)}[e]{c}{
    ($\forall s$: string)(
      $s \in c \Longleftrightarrow $ (
        def?($s$, $e$) $\yluego$ obtener($s$, $e$) = true
      )
    )
  }

%%%%%%%%%%%%%%%%%%%%%%%%%%

\subsection{Algoritmos}

\lstset{style=alg}

\begin{lstlisting}[mathescape]
'\alg{iVacio}{}{estr}'

    res $\leftarrow$ CrearDicc() '\ote{1}'

'\ofi{O(1)}'
\end{lstlisting}

\begin{lstlisting}[mathescape]
'\alg{iAgregar}{\In{s}{string}, \Inout{e}{estr}}{}'

    Definir(s, true, e)  '\ote{|s|}'

'\ofi{O(|s|)}'
\end{lstlisting}

\begin{lstlisting}[mathescape]
'\alg{iBorrar}{\In{s}{string}, \Inout{e}{estr}}{}'

    Definir(s, false, e)  '\ote{|s|}'

'\ofi{O(|s|)}'
\end{lstlisting}

\begin{lstlisting}[mathescape]
'\alg{iPertenece?}{\In{s}{string}, \Inout{e}{estr}}{bool}'

    res $\leftarrow$ Definido?(s, e) $\yluego$ Obtener(s, e) = true '\ote{|s|}'

'\ofi{O(|s|)}'
\end{lstlisting}


\pagebreak
\section{M�dulo Cola de Prioridad($\alpha$)}
\subsection{Interfaz}

  \textbf{se explica con}: Cola de Prioridad($\alpha$), Iterador Unidireccional Modificable($\alpha$)

  \textbf{usa}: Nat, bool
  
  \textbf{genero}: colaPrior($\alpha$), itColaPrior($\alpha$)
  
\subsubsection{Operaciones de Cola de Prioridad}

  \InterfazFuncion{Vacia}{}{colaPrior($\alpha$)}
  [true]
  {$res$ $\igobs$ vacia}
  [O(1)]
  [Crea una cola de prioridad]\\ 
  
  \InterfazFuncion{Vacia?}{\In{c}{colaPrior($\alpha$)}}{bool}
  [true]
  {$res$ $\igobs$ vacia?(c)}
  [O(1)]
  [Dice si la cola no tiene ningun elemento]\\ 

  \InterfazFuncion{Desencolar}{\Inout{c}{colaPrior($\alpha$)}}{$\alpha$}
  [$\neg$vacia?($c$) $\land$ $c$ $\igobs$ $c_0$]
  {$res$ $\igobs$ proximo($c_0$) $\land$ $c$ $\igobs$ desencolar($c_0$)}
  [O(log(tamano(c)))]
  [Quita el elemento mas prioritario]\\   
  
  \InterfazFuncion{Encolar}{\Inout{c}{colaPrior($\alpha$)}, \In{a}{$\alpha$}}{itColaPrior($\alpha$)}
  [$c$ $\igobs$ $c_0$ $\land$ $\neg$esta($a$, $c$)] %agregar el esta
  {$c$ $\igobs$ encolar(a,$c_0$) $\land$ Actual($res$) $\igobs$ $a$}
  [O(log(tamano(c)))]
  [Agrega al elemento a a la cola de prioridad]
  [El iterador se invalida si, y solo si se elimina el elemento siguiente del iterador sin llamar a la funcion Eliminar del mismo]\\ 


  \subsubsection{Operaciones auxiliares del TAD}
  \tadOperacion{tamano}{colaPrior($\alpha$)}{nat}{}
  \tadAxioma{tamano($c$)}{
    \IF vacia?($c$) THEN 0 ELSE 1 + tamano(desencolar($c$)) FI
  }

  ~ 

\subsubsection{Operaciones del iterador}

\InterfazFuncion{CrearIt}{\In{c}{colaPrior($\alpha$)}}{itColaPrior($\alpha$)}
  [true]
  {Siguientes($res$) $\igobs$ <> $\land$ Anteriores($res$) $\igobs$ <>}
  [O(1)]
  [Crea un iterador]
  [El iterador se invalida si, y solo si se elimina el elemento siguiente del iterador sin llamar a la funcion Eliminar del mismo]\\ 


\InterfazFuncion{Eliminar}{\Inout{it}{itColaPrior($\alpha$)}}{}
[$it$ $\igobs$ $it_0$ $\land$ HayMas?($it$)]
{$it$ $\igobs$ Eliminar($it_0$)}
[O(1)]
[elimina el elemento siguiente del iterador]\\ 

\InterfazFuncion{HayMas?}{\In{it}{itColaPrior($\alpha$)}}{bool}
[true]
{$res$ $\igobs$ HayMas?($it$)}
[O(1)]
[Dice si hay siguiente]\\ 

\InterfazFuncion{Actual}{\In{it}{itColaPrior($\alpha$)}}{$\alpha$}
[HayMas?($it$)]
{$res$ $\igobs$ Actual($it$)}
[O(1)]
[Devuelve el elemento actual]\\ 

\subsection{Representaci\'on}
  
  \begin{Estructura}{colaPrior($\alpha$)}[estr]

  \begin{Tupla}[estr]
    \tupItem{tam}{nat}%
    \tupItem{\\ cabeza}{puntero(nodo)}%
  \end{Tupla}

  ~

  \begin{Tupla}[nodo]
    \tupItem{padre}{puntero(nodo)}%
    \tupItem{\\ izq}{puntero(nodo)}%
    \tupItem{\\ der}{puntero(nodo)}%
    \tupItem{\\ dato}{puntero($\alpha$)}%
  \end{Tupla}

  \end{Estructura}  

  \subsubsection{Invariante de representaci\'on}

  \renewcommand{\labelenumi}{(\Roman{enumi})}

\begin{enumerate}
  \item Arbol Binario perfectamente balanceado
  \item El elemento con mayor prioridad se encuentra en el primer elemento
  \item Los hijos de un nodo son menores a su padre
  \item Es izquierdista, o sea, el \'ultimo nivel est\'a lleno desde la izquiera
  \item Sea n un nodo, solo su padre tiene a n como hijo y adem\'as, su padre no puede ser a la vez parte de su descendencia (no hay ciclos)
  \item El tam es igual a la cantidad de nodos de la cola, definido por la operacion tamano.

\end{enumerate} 


\tadOperacion{ColaALista}{c : puntero(nodo)}{secu(puntero(nodo))}{}
 \tadAxioma{ColaALista($c$)}{
   ( \IF $c$.izq == NULL THEN <> ELSE ColaALista(c.izq) FI $\&$  \IF $c$.der == NULL THEN <> ELSE ColaALista(c.der) FI ) $\circ$ c
  }

  \subsubsection{Funci\'on de abstracci\'on}
 
  %\AbsFc[acuerdoEstr]{}
  Abs : estr e $\rightarrow$ colaPrior($\alpha$) \{Rep(e)\} \\
  ($\forall$ e:estr) abs(e) $\equiv$ colaPrior | vacia?(colaPrior) $\equiv$ vacia?(e) $\land$ proximo(colaPrior) $\equiv$ desencolar(e) $\land$ encolar(colaPrior) $\equiv$ encolar(e)
 
  \Abs[estr]{colaPrior($\alpha$)}[e]{c}{
    vacia?($c$) $\igobs$ $e$.cabeza = NULL $\yluego$ \\
    $\neg$vacia?($c$) $\impluego$ ( \\
      proximo($c$) $\igobs$ *((*$e$.cabeza).dato) $\land$ \\
      desencolar($c$) $\igobs$ unir(Abs(<Tamano(*((*$e$.cabeza).izq)),*((*$e$.cabeza).izq)>), \\
       Abs(<Tamano(*((*$e$.cabeza).der)),*((*$e$.cabeza).der)>))
    )
  }
  ~
  ~ 
  \tadOperacion{unir}{colaPrior($\alpha$)/c1, colaPrior($\alpha$)/c2}{colaPrior($\alpha$)}{}
  \tadAxioma{unir($c1, c2$)}{
    \IF vacia?($c1$) THEN vacia ELSE unir(encolar(proximo($c2$), $c1$), desencolar($c2$)) FI
  }
  ~ 
  \tadOperacion{Tamano}{c : puntero(nodo)}{Nat)}{}
 \tadAxioma{Tamano($c$)}{
    1 + \IF $c$.izq == NULL THEN 0 ELSE Tamano(c.izq) FI + \IF $c$.der == NULL THEN 0 ELSE Tamano(c.der) FI 
  }
    ~ 


  \subsection{Representaci\'on del iterador}
  
  \begin{Estructura}{itColaPrior($\alpha$)}[iter]
    \begin{Tupla}[iter]
    \tupItem{siguiente}{puntero(nodo)}%
    \tupItem{cola}{puntero(estr)}%
  \end{Tupla} \\
  \end{Estructura}  

  \subsubsection{Invariante de Representaci\'on del iterador}

  Rep: iter $\rightarrow$ bool \\ 
  Rep(it) $\equiv$ true $\Longleftrightarrow$ Rep(*(it.cola)) $\land_L$ (it.siguiente = NULL $\lor_L$ esta?(it.siguiente,ColaALista(it.cola.cabeza) ) \\ 

  %esta mal
\subsubsection{Funci\'on de Abstracci\'on del iterador}
 
  %\AbsFc[acuerdoEstr]{}
  Abs : iter it $\rightarrow$ itMod($\alpha$)\\ 
     Abs(it) $\igobs$ m : itMod($\alpha$) | siguientes(m) $\equiv$ Abs(<Tamano(*(it.siguiente)),*(it.siguiente)>) $\bullet$ <> $\land$ \\
      anteriores(m) $\equiv$ <>
  %esta mal

\subsection{Algoritmos}
\subsubsection{Algoritmos de Cola de Prioridad}

\lstset{style=alg}

\begin{lstlisting}[mathescape]
'\alg{iVacia}{\In{n}{nat}}{colaPrior}'
    
    res.tam $\leftarrow$ 0 '\ote{1}'
    res.cabeza $\leftarrow$ NULL    '\ote{1}'
   
    
'\ofi{O(1)}'
\end{lstlisting}

\begin{lstlisting}[mathescape]
'\alg{iVacia?}{\In{c}{estr}}{bool}'
    
    if c.cabeza = NULL: '\ote{1}'
      res $\leftarrow$ true    '\ote{1}'
    else:
      res $\leftarrow$ false  '\ote{1}'
    endif    
    
'\ofi{O(1)}'
\end{lstlisting}

\begin{lstlisting}[mathescape]
'\alg{iDesencolar}{\Inout c: estr)}{$\alpha$}'

    puntero(nodo) padreUltimo, ultimo, primerIzquierda, primerDerecha, izquierda, derecha \\
$\alpha$ dato , maximo \\
res  $\leftarrow$ c.cabeza.dato \\
if c.tamano > 1 :                                         '\ote{1}'
    padreUltimo $\leftarrow$ damePadre(c.tam -1)         '\ote{log(c.tam -1)}'
    if padreUltimo.der = NULL :                                         '\ote{1}'
        ultimo $\leftarrow$ padreUltimo.izq                             '\ote{1}'
        padreUltimo.izq $\leftarrow$ NULL                               '\ote{1}'
    else: \\
        ultimo $\leftarrow$ padreUltimo.der                             '\ote{1}'
        padreUltimo.der $\leftarrow$ NULL                                   '\ote{1}'
    endif    
    ultimo.padre $\leftarrow$ NULL                                      '\ote{1}'
    primerIzquierda  $\leftarrow$ c.cabeza.izq                          '\ote{1}'
    ultimo.izq $\leftarrow$ primerIzquierda                             '\ote{1}'
    if primerIzquierda != NULL :                                        '\ote{1}'
        primerIzquierda.padre = ultimo                                  '\ote{1}'
    endif    
    primerDerecha  $\leftarrow$ c.cabeza.der                            '\ote{1}'
    ultimo.der  $\leftarrow$ primerDerecha                              '\ote{1}'
    if primerDerecha != NULL :                                          '\ote{1}'
        primerDerecha.padre  $\leftarrow$ ultimo                        '\ote{1}'
    endif    
    c.cabeza  $\leftarrow$ ultimo                                       '\ote{1}'
    bool inPlace  $\leftarrow$ false                                 '\ote{1}'
    dato  $\leftarrow$ ultimo.dato                                      '\ote{1}'
    while $\neg$inPlace :                                               '\ote{log(c.tam)}'
        izquierda  $\leftarrow$ ultimo.izq                              '\ote{1}'
        derecha  $\leftarrow$ ultimo.der                                '\ote{1}'
        if derecha != NULL :                                            '\ote{1}'
            maximo  $\leftarrow$ max(izquierda.v, derecha.v)            '\ote{1}'
            if maximo < dato :                                          '\ote{1}'
                inPlace  $\leftarrow$ true                              '\ote{1}'
            else: 
                if maximo = izquierda.dato :                            '\ote{1}'
                    swapConPadre(izquierda)                             '\ote{1}'
                else: 
                    swapConPadre(derecha)                               '\ote{1}'
                endif
            endif        
        else:
            if izquierda = NULL :                                       '\ote{1}'
                inPlace  $\leftarrow$ true                              '\ote{1}'
            else:
                if izquierda.dato > dato :                              '\ote{1}'
                    swapConPadre(izquierda)                             '\ote{1}'
                else: 
                    inPlace  $\leftarrow$ true                          '\ote{1}'
                endif
            endif
        endif            
else:\\
    c.cabeza  $\leftarrow$ NULL                                         '\ote{1}'
c.tam  $\leftarrow$ c.tam -1                                            '\ote{1}'

'\ofi{O(log(c.tam-1) + 17*O(1) + O(log(c.tam)*14) + 2*O(1)) = O(log(c.tam))}'
\end{lstlisting}


\begin{lstlisting}[mathescape]
'\alg{iEncolar}{\Inout c: estr, \In dato: $\alpha$}{itColaPrior($\alpha$)}'

    puntero(nodo) nodo = iAgregarAtras(c, dato)         '\ote{1}'
    iLevantar(c, nodo, false)        '\ote{log(c.tam)}'
    iter it  $\leftarrow$ iCrearIt(c)  '\ote{1}'
    it.siguiente $\leftarrow$ nodo '\ote{1}'
    res $\leftarrow$  it '\ote{1}'
'\ofi{O(log(c.tam))}'
\end{lstlisting}


\begin{lstlisting}[mathescape]
pre '$\equiv$' {c.tam > 1 && 0 >= posicion < tam}
post '$\equiv$' {me devuelve el padre de la posicion que le pase}
'\alg{iDamePadre}{\Inout c: estr, \In posicion: Nat)}{puntero(nodo)}'

  Nat arraySize$\leftarrow$ 0                                    '\ote{1}'
  Nat tam$\leftarrow$ posicion                                    '\ote{1}'
  while tam > 0:                                              '\ote{log(posicion)}'
      arraySize$\leftarrow$ arraySize + 1                   '\ote{1}'
      tam$\leftarrow$ (tam - 1)/2                             '\ote{1}'
  Array(arraySize) array                                  '\ote{1}'

  //arraySize queda de tama'\~n'o log(posicion) 
  //y posicion es en el peor de los casos c.tam

  Nat cant  $\leftarrow$ 0                                '\ote{1}'
  while cant < arraySize:                                 '\ote{log(posicion)}'
      array[cant]  $\leftarrow$ -1                         '\ote{1}'
      cant $\leftarrow$ cant + 1  '\ote{1}'

  Nat index$\leftarrow$ arraySize - 1                            '\ote{1}'
  tam$\leftarrow$ posicion                                    '\ote{1}'

  while tam > 0:                                              '\ote{log(posicion)}'
      array[index]$\leftarrow$ tam mod 2                      '\ote{1}'
      index$\leftarrow$ index - 1                             '\ote{1}'
      tam = (tam - 1)/2                                       '\ote{1}'

  index$\leftarrow$ 0                                         '\ote{1}'
  puntero(nodo) actual$\leftarrow$ c.cabeza                                   '\ote{1}'

  while index < arraySize - 1:                               '\ote{log(posicion)}'
      if array[index] = 0:                                    '\ote{1}'
          actual$\leftarrow$ actual.der                       '\ote{1}'
      else:
          actual$\leftarrow$ actual.izq                       '\ote{1}'
      endif
      index$\leftarrow$ index + 1                             '\ote{1}'
      
  return actual         '\ote{1}'

'\ofi{2*O(1) + O(log(posicion))*2*O(1) + O(log(posicion))*O(1) + 2*O(1) + O(log(posicion))*3*O(1) + 2*O(1) + O(log(posicion))*4*O(1) + O(1) = O(log(posicion) = O(log(c.tam))}'
\end{lstlisting}

\begin{lstlisting}[mathescape]
pre '$\equiv$' {el nodo esta dentro del arbol}
post '$\equiv$' {me devuelve el arbol con el dato del nodo que le pase en la posicion '\\ '
 de su padre y el dato del padre en la posicion del nodo original}
'\alg{iSwapConPadre}{\Inout c: estr, \In nodo: puntero(nodo)}{}'

  puntero(nodo) padreObj$\leftarrow$ nodo.padre                             '\ote{1}'
  puntero(nodo) abuelo$\leftarrow$ padreObj.padre                           '\ote{1}'
  Bool hermanoADer$\leftarrow$ padreObj.izq.dato = nodo.dato     '\ote{1}'
  puntero(nodo) hermano
  if hermanoADer:
      hermano$\leftarrow$ padreObj.der                        '\ote{1}'
  else:
      hermano$\leftarrow$ padreObj.izq                        '\ote{1}'
  endif
  
  puntero(nodo) hijoIzq$\leftarrow$ nodo.izq                                '\ote{1}'
  puntero(nodo) hijoDer$\leftarrow$ nodo.der                                '\ote{1}'
  
  if abuelo != NULL
      if(abuelo.izq.dato = padreObj.dato):
          abuelo.izq $\leftarrow$ nodo                          '\ote{1}'
      else:
          abuelo.der $\leftarrow$ nodo                          '\ote{1}'
  endif
  
  if hijoIzq != NULL
      hijoIzq.padre$\leftarrow$ padreObj                     '\ote{1}'
  endif
  
  if hijoDer != NULL)
      hijoDer.padre $\leftarrow$ padreObj                     '\ote{1}'
  endif
                              
  padreObj.izq$\leftarrow$ hijoIzq                           '\ote{1}'
  padreObj.der$\leftarrow$ hijoDer                           '\ote{1}'
  padreObj.padre$\leftarrow$ nodo                             '\ote{1}'
  
  if hermano != NULL
      hermano.padre$\leftarrow$ nodo                           '\ote{1}'
  endif
  
  nodo.padre $\leftarrow$ abuelo                                '\ote{1}'
  
  if (hermanoADer)
      nodo.der$\leftarrow$ hermano                             '\ote{1}'
      nodo.izq$\leftarrow$ padreObj                           '\ote{1}'
  else:
      nodo.izq$\leftarrow$ hermano                             '\ote{1}'
      nodo.der$\leftarrow$ padreObj                           '\ote{1}'
  endif
  
  if (nodo.padre = NULL)
      c.cabeza $\leftarrow$ nodo                                   '\ote{1}'
  endif

'\ofi{21*O(1) = O(1)}'
\end{lstlisting}


\begin{lstlisting}[mathescape]
pre '$\equiv$' {el nodo a levantar se encuentra en el arbol}
post '$\equiv$' {si forceLevantar es true devuelve el arbol con el dato del '\\ '
 nodo en la primer posicion del arbol y si no balancea el arbol}
'\alg{iLevantar}{\Inout c: estr, \In nodo: puntero(nodo), \In forceLevantar: bool}{}'

    while nodo.padre != NULL $\land$ (nodo.dato > nodo.padre $\lor$ forceLevantar):      '\ote{log(c.tam)}'
          iSwapConPadre(c,nodo)          '\ote{1}'

'\ofi{O(log(c.tam))}'
\end{lstlisting}

\begin{lstlisting}[mathescape]
'\alg{iAgregarAtras}{\Inout c: estr, \In d: $\alpha$}{puntero(nodo)}'

puntero(nodo) actual                                  '\ote{1}'
puntero(nodo) nuevo                                   '\ote{1}'
if c.tam = 0:                                      '\ote{1}'
    actual $\leftarrow$ NULL                       '\ote{1}'
else:
    actual $\leftarrow$ iDamePadre(c.tam)          '\ote{1}'
    nuevo.dato $\leftarrow$ d                      '\ote{1}'
    nuevo.padre  $\leftarrow$ actual               '\ote{1}'
if c.tam = 0:                                      '\ote{1}'
    c.cabeza$\leftarrow$ nuevo                     '\ote{1}'
else:
    if c.tam % 2 == 0:                         '\ote{1}'
        actual.der = nuevo                    '\ote{1}'
    else:
        actual.izq = nuevo                    '\ote{1}'
c.tam = c.tam + 1                                      '\ote{1}'
ret  $\leftarrow$  nuevo                               '\ote{1}'

'\ofi{O(11*O(1)) = O(1)}'
\end{lstlisting}

\subsubsection{Algoritmos del iterador}

\begin{lstlisting}[mathescape]
'\alg{iCrearIt}{c : estr}{iter}'

iter iterador '\ote{1}'
iterador.cola $\leftarrow$  c '\ote{1}'
iterador.siguiente $\leftarrow$ NULL '\ote{1}'
res $\leftarrow$ iterador '\ote{1}'

'\ofi{O(1)}'
\end{lstlisting}

\begin{lstlisting}[mathescape]
'\alg{iHayMas?}{it : iter}{bool}'

if it.siguiente != NULL '\ote{1}'
   res $\leftarrow$ true '\ote{1}'
else
   res $\leftarrow$ false '\ote{1}'   
endif

'\ofi{O(1)}'
\end{lstlisting}

\begin{lstlisting}[mathescape]
'\alg{iEliminar}{it : iter}{}'

iLevantar(it.cola,it.siguiente,true) '\ote{log(c.tam)}'   
iDesencolar(it.cola) '\ote{log(c.tam)}'   
it.siguiente $\leftarrow$  NULL'\ote{1}'

'\ofi{O(2*log(c.tam))}'
\end{lstlisting}

\begin{lstlisting}[mathescape]
'\alg{iActual}{it : iter}{$\alpha$}'

res $\leftarrow$ *(it.siguiente)

'\ofi{O(1)}'
\end{lstlisting}





\pagebreak
\section{M\'odulo Restriccion}
\subsection{Interfaz}

\textbf{se explica con}: Restriccion

  \textbf{usa}: tags, $dicc_t$(tags,bool), bool
  
  \textbf{genero}: genero
  
\subsubsection{Operaciones de Restriccion}

\InterfazFuncion{CrearRestriccion}{\In{t}{string}}{restr}
  [t != AND $\wedge$ t != OR $\wedge$ t != NOT $\wedge$ 0 < $|$t$|$ < 65]
  {$res \igobs$ $<t>$}
  [O($|$r$|$)]
  [Crea una nueva restriccion en base a un tag unico] \\

\InterfazFuncion{Negar}{\In{r}{restr}}{restr}
  [r $\igobs$ $r_0$]
  {r $\igobs$ NOT $r_0$}
  [O($|$r$|$)]
  [Niega la restriccion dada, generando otra] \\

\InterfazFuncion{And}{\In {r}{restr}, \In {rOtra}{restr}}{restr}
  [r $\igobs$ $r_0$]
  {r $\igobs$ $r_0$ AND rOtra}
  [O($|$r$|$)]
  [Hace and de las dos restricciones, generando otra.] \\ 

\InterfazFuncion{Or}{\In {r}{restr}, \In {rOtra}{restr}}{restr}
  [r $\igobs$ $r_0$]
  {r $\igobs$ $r_0$ Or rOtra}
  [O($|$r$|$)]
  [Hace or de las dos restricciones, generando otra.] \\

  \InterfazFuncion{Copiar}{\In {r}{restr}}{restr}
  [true]
  {res $\igobs$ r}
  [O($|$r$|$)]
  [Crea una copia de la restricci\'on] \\

\InterfazFuncion{Verifica}{\In {r}{restr},\In {ts}{$dicc_t$(tags,bool)}}{bool}
  [True]
  {res == verifica?(ts,r)}
  [O($|$r$|$)]

\subsection{Representaci\'on}
\subsubsection{Representaci\'on de Restricci\'on}

Representamos cada nodo del arbol con una clave y dos punteros a nodos hijos. Si la clave esta entre los strings (AND, OR, NOT), se considera al nodo como un nodo de operacion, y por lo tanto, si la clave es AND u OR sus dos punteros no pueden ser NULL, y si la clave es NOT, su puntero derecho debe ser NULL y su puntero izquiero no. Los nodos cuya clave no este entre las palabras (AND, OR y NOT) deberan tener como NULL a sus dos punteros.

\begin{Estructura}{restr}[restr\_tup]
    \begin{Tupla}[restr\_tup]
      \tupItem{nombre}{string}
      \tupItem{izq}{puntero(restr)}
      \tupItem{der}{puntero(restr)}
      \tupItem{tam}{nat}
    \end{Tupla}
    \end{Estructura} 
    
\subsubsection{Invariante de Representaci\'on}

\renewcommand{\labelenumi}{(\Roman{enumi})}
\begin{enumerate}
\item El tama\~no de un nodo se define como la suma del tama\~no de sus hijos + 1.
\item El arbol no debe tener ciclos, o sea, existe n en Nat tal que n > al tama\~no de cada uno de los nodos del \'arbol
\item Los nodos hoja tienen clave disinta de (AND, OR, NOT)
\item Los nodos NO hoja tienen su clave entre los valores (AND, OR, NOT)
\item Si el nodo tiene clave entre los valores (AND, OR) sus dos punteros son distintos de NULL
\item Si el nodo tiene como clave NOT, su puntero derecho debera ser NULL y su puntero izquierdo debera ser distinto de NULL
\item Cada nodo es un arbol en si mismo
\end{enumerate} 
    
\subsubsection{Funci\'on de Abstracci\'on}
Abs : restr\_tup  $\rightarrow$ Restriccion \\

Abs(e) $\igobs$ r: restr\_tup \  |\  $\forall$ cj : conj(tag) \\
\phantom{pala}Verifica?(r, cj) $\Leftrightarrow$ Verifica$?$(e, cj)


\subsection{Algoritmos}

\lstset{style=alg}

\begin{lstlisting}[mathescape]
'\alg{iCrearRestriccion}{\In{t}{string}}{restr}'
    
    restr restriccion
    restriccion.nombre $\leftarrow$ t '\ote{1}'
    restriccion.izq $\leftarrow$ NULL '\ote{1}'
    restriccion.der $\leftarrow$ NULL '\ote{1}'
    restriccion.tam $\leftarrow$ 1 '\ote{1}'
    res $\leftarrow$ *restriccion  '\ote{1}'
    
'\ofi{O(1)}'
\end{lstlisting}

\begin{lstlisting}[mathescape]
'\alg{iNegar}{\Inout{r}{restr}}{restr}'
    
    restr not $\rightarrow$  <"NOT",NULL,NULL,NULL> '\ote{1}'
    not$\rightarrow$izq  $\leftarrow$ iCopiar(r) '\ote{r.tam}'
    not.tam  $\leftarrow$ r.tam + 1 '\ote{1}'
    ret $\leftarrow$ not '\ote{1}'
   
'\ofi{O(r.tam)}'
\end{lstlisting}

\begin{lstlisting}[mathescape]
'\alg{iOr}{\In{r}{restr}, \In{rOtra}{restr}}{restr}'

    restr or $\rightarrow$  <"OR",NULL,NULL,NULL> '\ote{1}'
    or$\rightarrow$izq  $\leftarrow$ iCopiar(r) '\ote{r.tam}'
    or$\rightarrow$ der  $\leftarrow$ iCopiar(rOtra) '\ote{r.tam}'
    or.tam  $\leftarrow$ r.tam + rOtra.tam + 1 '\ote{1}'
    ret $\leftarrow$ or '\ote{1}'
    
'\ofi{O(r.tam)}'
\end{lstlisting}

\begin{lstlisting}[mathescape]
'\alg{iAnd}{\Inout {r}{restr}, \In {rOtra}{restr}}{restr}'
    
    estr and $\rightarrow$  <"AND",NULL,NULL,NULL> '\ote{1}'
    and$\rightarrow$izq  $\leftarrow$ iCopiar(r) '\ote{r.tam}'
    and$\rightarrow$ der  $\leftarrow$ iCopiar(rOtra) '\ote{r.tam}'
    and.tam  $\leftarrow$ r.tam + rOtra.tam + 1 '\ote{1}'
    ret $\leftarrow$ and '\ote{1}'
    
'\ofi{O(r.tam)}'
\end{lstlisting}

\begin{lstlisting}[mathescape]
'\alg{iVerifica}{\In {r}{restr},\In {ts}{$dicc_t$(tags,bool)}}{bool}'
    
   if r.nombre = "NOT" : '\ote{1}'
     res $\leftarrow$ ($\neg$ iVerifica(r.izq,ts)) '\ote{r.tam}'
   else if r.nombre  = "AND": '\ote{1}'
     res $\leftarrow$ (iVerifica(r.izq,ts) && iVerifica(r.der,ts)) '\ote{r.tam}'
   else if r.nombre = "OR": '\ote{1}'
     res $\leftarrow$ (iVerifica(r.izq,ts) || iVerifica(r.der,ts)) '\ote{r.tam}'
   else:
    res $\leftarrow$ definido?(r.nombre, ts)
   endif
       
'\ofi{O(r.tam)}'
\end{lstlisting}

\subsubsection{Complejidad de iVerifica}
Calcularemos la complejidad en base a m, la cantidad de nodos del \'arbol. 
En el peor caso de iVerifica (AND o OR) nos quedamos con dos subproblemas que tienen la mitad de tama\~no que el original, as\'i que
a = 2 y c = 2.
M\'as all\'a de esto, se hace una cantidad constante de comparaciones y asignaciones, por lo que podemos decir que f(n) $\in$ O(1).
Usamos el primer caso del Teorema Maestro, porque $n^{log_2(2)-\epsilon}$ = n$^{1-\epsilon}$. Entonces, si elegimos, por ejemplo
$\epsilon$ = 1, nos queda O($n^0$) = O(1) y sabemos que f(n) tiene esa complejidad.
Por lo tanto, utilizando el primer caso tenemos que T(n) $\in$ O($m^{log_2(2)}$) = O(m).

\begin{lstlisting}[mathescape]
'\alg{iCopiar}{\In {r}{restr}}{restr}'
    
   restr copia
   copia.nombre  $\leftarrow$ r.nombre '\ote{64}'
   copia.tam  $\leftarrow$ r.tam '\ote{1}'
   copia.izq  $\leftarrow$ NULL '\ote{1}'
   copia.der  $\leftarrow$ NULL '\ote{1}'
   if r.izq != NULL
    copia.izq  $\leftarrow$  iCopiar(&(r.izq)) '\ote{r.izq$\rightarrow$tam}'
   end if
   if r.der != NULL
    copia.der  $\leftarrow$  iCopiar(&(r.der)) '\ote{copy(r.der$\rightarrow$tam)}'
   end if 
       
'\ofi{O(r.tam)}'
\end{lstlisting}

\subsubsection{Complejidad de iCopiar}
La demostraci\'on de la complejidad es an\'aloga a la de iVerifica ya que es la misma recursi\'on y el caso base es O(1).





\pagebreak
 
\section{Modulo Ciudad}


\subsection{Interfaz}

  \textbf{se explica con}: \tadNombre{Ciudad}, \tadNombre{Iterador Unidireccional(nat)}.

  \textbf{generos}: \TipoVariable{ciudad} \TipoVariable{itVectorNat}.

 % \textbf{usa}: Vector($\alpha$), Bool, Nat, Mapa, Cola de Prioridad($\alpha$), Conjunto de String, DiccionarioString($\alpha$)

  %\textbf{exporta}:

  \subsubsection{Operaciones de Ciudad}

  \InterfazFuncion{Crear}{\In{m}{mapa}}{ciudad}
  [true] % invRep(mapa) o algo como mapa valido ??
  {$res$ $\igobs$ crear($m$)}
  [$O(1)$]
  [crea una ciudad robotica con un mapa preestablecido]
  [el mapa se agrega a la ciudad por referencia y no se puede modificar]\\ 

  \InterfazFuncion {Entrar}{\In{cs}{conj$_T$(string)}, \In{e}{string}, \Inout{c}{ciudad}}{}
  [$e$ $\in$ estaciones(mapa($c$)) $\land$ $c$ $\igobs$ $c_0$]
  {$c \igobs$ entrar($cs$, $e$, $c_0$)}
  [$O$($|e_m|$ * E + ($|e_m|$ + R) * S + $N_{total}$)]
  [agrega un robot a la ciudad, su rur es la cantidad de robots ya agregados, sus tags y estacion son los parametros pasados]
  [se pasa todo por referencia] \\  

  \InterfazFuncion{Mover}{\In{u}{nat}, \In{e}{string}, \Inout{c}{ciudad}}{}
  [$e \in$ estaciones(mapa($c$)) $\land$
  $u \in$ robots($c$) $\land$ 
  conectadas?($e$, estacion($u$), $c$) $\land$
  $c \igobs c_0$]
  {$c \igobs$ mover($u$, $e$, $c_0$)}
  [$O(|e1| + |e2| + log N_{e1} + log N_{e2} )$] %ver
  [mueve un robot de una estacion a otra]
  [] \\  

  \InterfazFuncion{Inspeccion}{\In{e}{string}, \Inout{c}{ciudad}}{}
  [$e \in$ estaciones(mapa($c$)) $\land$ $c$ $\igobs$ $c_0$]
  {$c \igobs$ inspeccion($e$, $c_0$)}
  [$O(|e| + log N_e)$]
  [realiza la inspeccion, eliminando si es necesario el robot mas infractor en la estacion]
  [] \\  

  \InterfazFuncion{Robots}{\In{c}{ciudad}}{itVectorNat}
  [true]
  {esPermutacion?(Siguientes($res$), conjALista(robots($c$)))}
  [$O(1)$]
  [dada una ciudad, me da un iterador de sus rurs]
  [] \\  

  \InterfazFuncion {Estaciones}{\In {c}{ciudad}} {itLista(string)}
  [true]
  {SecuSuby($res$) $\igobs$ estaciones(mapa($c$))}
  [$O(1)$]
  [Dada una ciudad, me da un iterador de sus rurs]
  [] \\  

  \InterfazFuncion {Infracciones} {\In{n}{nat}, \In{c}{ciudad}}{nat}
  [$n$ $\in$ robots($c$)]
  {$res \igobs$ \#infracciones($n$, $c$)}
  [$O(1)$]
  [Dado un robot, me da sus infracciones]
  [] \\  

  \InterfazFuncion {Estacion}{\In{n}{nat}, \In{c}{ciudad}}{string}
  [$n \in$ robots($c$)]
  {$res \igobs$ estacion($n$, $c$)}
  [$O(1)$]
  [Dado un robot, me da su estacion actual]
  [] \\  

  \InterfazFuncion {Tags}{\In{n}{nat}, \In{c}{ciudad}}{$conj_{T}$}
  [$n \in$ robots($c$)]
  {$res \igobs$ tags($n$, $c$)}
  [$O(1)$]
  [Dado un robot, me da sus caracteristicas]
  [] \\  

  \subsubsection{Operaciones de itVectorNat}

  \InterfazFuncion{CrearIt}{\In{c}{Ciudad}}{itVectorNat}
  [true]
  {Siguientes($res$) $\igobs$ $indices(vivos(c))$}
  [$O(1)$]
  [Crea y devuelve un iterador de los robots.]\\

  \InterfazFuncion{HayMas?}{\In{it}{itVectorNat}}{bool}
  [true]
  {$res$ $\igobs$ HayMas?($it$)}
  [$O(Longitud(Siguientes(it))$]
  [Informa si hay m\'as elementos por iterar.]\\
  
  \InterfazFuncion{Pr\'oximo}{\Inout{it}{itVectorNat}}{nat}
  [HayMas?(it) $\land$ $it \igobs it_0$]
  {$it$ $\igobs$ Avanzar($it_0$) $\land$ $res \igobs$ Actual($it_0$)}
  [$O(1)$]
  [Avanza el iterador y devuelve el actual.]\\

\subsection{Representaci\'on}

  \begin{Estructura}{ciudad}[estr]
    \begin{Tupla}[estr]
      \tupItem{\#robots}{nat}%
      \tupItem{\\ \#infracciones}{vector(nat)}%
      \tupItem{\\ estacion}{vector(string)}%
      \tupItem{\\ tags}{vector(conj$_T$)}%
      \tupItem{\\ vivos}{vector(bool)}%
      \tupItem{\\ sendasPermitidas}{vector(dicc$_T$(dicc$_T$(bool)))}%
      \tupItem{\\ robotsEnEstacion}{dicc$_T$(colaPrior(tupla(nat, nat)))}%
      \tupItem{\\ sacarRobotDeEstacion}{vector(itColaPrior)}%
      \tupItem{\\ mapa}{map}%
    \end{Tupla}

  \end{Estructura}

\subsubsection{Invariante de Representaci\'on}

\renewcommand{\labelenumi}{(\Roman{enumi})}

\begin{enumerate}
\item Las longitudes de los vectores coinciden con la cantidad de robots
\item Todas las estaciones estan en las estaciones del mapa de la ciudad
\item Las estaciones padres en las sendas tienen orden lexicograficamente mayor a sus estaciones conectadas
\item Para cada robot vivo, el siguiente de su iterador en sacarRobotDeEstacion es una tupla cuyo segundo valor es el rur
\item Para cada robot vivo, su estacion en la posicion correspondiente del vector esta dentro del mapa
\item Para cada robot vivo, sus sendasPermitidas corresponden a combinaciones de estaciones que estan conectadas segun el mapa
y cuya restriccion es verificada por los tags del robot
\item Los indices que son true de los vivos, son los mismos que combinar todos los rurs de robotsEnEstacion
\item Para cada robot de robotsEnEstacion, su estacion es la que indica el vector de estaciones y sus infracciones las que indica el vector de infracciones
\end{enumerate}

\Rep[estr][e]{

(long($e$.\#infracciones) = $e$.\#robots $\land$ long($e$.estacion) = $e$.\#robots $\land$ \\
long($e$.tags) = $e$.\#robots $\land$ long($e$.vivos) = $e$.\#robots $\land$ \\
long($e$.sendasPermitidas) = $e$.\#robots $\land$ long($e$.sacarRobotDeEstacion) = $e$.\#robots) \\ $\yluego$ \\

($\forall s$: string)(
  (def?($s$, $e$.sendasPermitidas) $\lor$ \\
  def?($s$, $e$.robotsEnEstacion)) \\
  $\Rightarrow$ esta($s$, estaciones($e$.mapa))
) \\ $\yluego$ \\

($\forall s, t$: string)(
  (def?($t$, $e$.sendasPermitidas) $\yluego$ def?($s$, obtener($t$, $e$.sendasPermitidas)))\\
  $\Rightarrow$ (esta($s$, estaciones($e$.mapa)) $\land$ $s$ < $t$)
) \\ $\yluego$ \\

($\forall i$: nat)(
  ($i$ < $e$.\#robots $\yluego$ $e$.vivos[$i$] = true) $\impluego$ \\
  (
    (HaySiguiente?(e.sacarRobotDeEstacion[i]) $\yluego$ $\pi_2$(Siguiente(e.sacarRobotDeEstacion[i])) = i) $\land$ \\
    esta($e$.estacion[$i$], estaciones($e$.mapa)) $\land$ \\
    (
      ($\forall s, t$: string)(\\
        (def?($s$, $e$.tags) $\Rightarrow$ 0 < long($s$) < 65) $\land$ \\
        (
          (def?($s$, $e$.sendasPermitidas[$i$]) $\yluego$ def?($t$, obtener($s$, $e$.sendasPermitidas[i]))) <=> \\
          (
            conectadas?($s$, $t$, $e$.mapa) $\yluego$ \\
            verifica?($e$.tags[$i$], restriccion($s, t, e$.mapa))
          )
        )
      )
    )
  )
) \\ $\land$ \\

esPermutacion?(indices($e$.vivos), combinar($e$.robotsEnEstacion)) \\ $\land$ \\

($\forall s$: string)(
  (def?($s$, $e$.robotsEnEstacion) $\impluego$ \\
  (
    ($\forall t$: tupla(string, string)) \\
    (
      (esta?($t$, inorder(obtener($s$, $e$.robotsEnEstacion)))) $\Rightarrow$ \\
      (
        $\pi_1(t)$ < $e$.\#robots $\yluego$ \\
        (
          $e$.estacion[$\pi_1(t)$] = s $\land$ \\
          $e$.\#infracciones[$\pi_1(t)$] = $\pi_2(t)$
        )
      )
    )
  )
)

}

\mbox{}

  ~     
  
  \tadOperacion{combinar}{dicc(string, colaPrior(tupla(nat, nat)))/d}{lista(nat)}{}
  \tadAxioma{combinar($d$)}{auxCombinar($d$, $d$.claves)}

  ~     
  
  \tadOperacion{auxCombinar}{dicc(string, colaPrior(tupla(nat, nat)))/d, lista(string)/l}{lista(nat)}{$ l \subseteq d$.claves}
  \tadAxioma{auxCombinar($d, l$)}{
    \IF vacia?($l$) THEN <> ELSE filtrar(colaALista(obtener(prim($l$), $d$))) \& auxCombinar($d$, fin($l$)) FI
  }

  ~     
  
  \tadOperacion{filtrar}{lista(tupla(nat, nat))/l}{lista(nat)}{}
  \tadAxioma{filtrar($l$)}{
    \IF vacia?($l$) THEN <> ELSE $\pi_2$(prim($l$)) \textbullet $ $ filtrar(fin($l$)) FI
  }

  ~     
  
  \tadOperacion{indices}{lista(bool)/l}{lista(nat)}{}
  \tadAxioma{indices($l$)}{indicesDesde($l$, 0)}

  ~     
  
  \tadOperacion{indicesDesde}{lista(bool)/l, nat/n}{lista(nat)}{}
  \tadAxioma{indicesDesde($l, n$)}{
    \IF vacia?($l$) THEN <> ELSE {
      \IF prim($l$) = true THEN
        $n $ \textbullet indicesDesde($l$, n+1) 
      ELSE
        indicesDesde($l$, n+1)
      FI
      }
    FI
  }

  ~     
  
  \tadOperacion{esPermutacion?}{lista($\alpha$)/l1, lista($\alpha$)/l2}{bool}{}
  \tadAxioma{esPermutacion?($l1, l2$)}{
    estan?($l1, l2$) $\land$ estan?($l2, l1$)
  }

  ~     
  
  \tadOperacion{estan?}{lista($\alpha$)/l1, lista($\alpha$)/l2}{bool}{}
  \tadAxioma{estan?($l1, l2$)}{
    vacia?($l2$) $\oluego$ ($\neg$vacia?($l1$) $\yluego$ (esta?(prim($l1$), $l2$) $\land$ estan?(fin($l1$), sacar(prim($l1$), $l2$))))
  }
  
  ~     
  
  \tadOperacion{sacar}{$\alpha$/e, lista($\alpha$)/l}{bool}{}
  \tadAxioma{sacar($e, l$)}{
    \IF vacia?($l$) THEN <> ELSE {
      \IF prim($l$ = $e$) THEN
        fin($l$)
      ELSE
        $e$ \textbullet $ $ sacar($e$, fin($l$))
      FI
    } FI
  }
  Obs: $\alpha$ debe ser comparable con la funcion =.


  \tadOperacion{conjALista}{conj($\alpha$)/c}{lista($\alpha$)}{}
  \tadAxioma{conjALista?($c$)}{
    \IF $\emptyset$?($c$) THEN <> ELSE dameUno($c$) $ $ \textbullet $ $ conjALista(sinUno($c$)) FI
  }

%termina el invRep

\subsubsection{Funci\'on de Abstracci\'on}
  
  \Abs[estr]{ciudad}[e]{c}{
    proximoRUR($c$) $\igobs$ $e$.\#robots $\land$ \\
    mapa($c$) $\igobs$ $e$.mapa $\land$ \\
    robots($c$) $\igobs$ listaAConj(indices($e$.vivos)) $\yluego$ \\
    ($\forall i$: nat)(
      ($i$ $\in$ robots($c$)) $\impluego$ \\ (
        estacion($i$, $c$) $\igobs$ $e$.estacion[$i$] $\land$ \\
        tags($i$, $c$) $\igobs$  $e$.tags[$i$] $\land$ \\
        \#infracciones($i$, $c$) $\igobs$  $e$.\#infracciones[$i$]
      )
    )
  }

  ~     

  Obs: indices fue previamente definido para el Rep.

  ~     
  
  \tadOperacion{listaAConj}{lista($\alpha$)/l}{conj($\alpha$)}{}
  \tadAxioma{listaAConj($l$)}{
    \IF vacia?($l$) THEN $\emptyset$ ELSE Ag(prim($l$), listaAConj(fin($l$))) FI
  }

  \subsubsection{Representaci\'on de itVectorNat}

  \begin{Estructura}{itVectorNat}[itVect]
    \begin{Tupla}[itVect]
      \tupItem{elementos}{vector(bool)}
      \tupItem{actual}{nat}%
    \end{Tupla}
  \end{Estructura}
  
  \subsubsection{Invariante de Representaci\'on de itVectorNat}
  
  ~

  \Rep[itVectorNat][it]{true}
  
  ~

  \subsubsection{Funcion abstracci\'on de itVectorNat}
  \Abs[itVectorNat]{itUni($nat$)}[it]{u}{Siguientes(u) $=$ Abs(indicesDesdeIndice(it.actual, it.elementos)}

  ~     
  
  \tadOperacion{indicesDesdeIndice}{nat/n, lista(bool)/l}{conj($\alpha$)}{}
  \tadAxioma{indicesDesdeIndice($n, l$)}{
    \IF vacia?($l$) THEN <> ELSE $n $ \textbullet $ $ indicesDesdeIndice(n+1, fin($l$)) FI
  }

%%%%%%%%%%%%%%%%%%%%%%

\subsection{Algoritmos}

  \subsubsection{Algoritmos de Ciudad}

  \lstset{style=alg}

   \begin{lstlisting}[mathescape]
'\alg{iCrear}{\In{m}{map}}{estr}'

  res.#robots $\leftarrow$ 0  '\ote{1}'
  res.#infracciones $\leftarrow$ Vacia()  '\ote{1}'
  res.estacion $\leftarrow$ Vacia()   '\ote{1}'
  res.tags $\leftarrow$ Vacia()   '\ote{1}'
  res.vivos $\leftarrow$ Vacia()  '\ote{1}'
  res.sacarRobotDeEstacion $\leftarrow$ Vacia()   '\ote{1}'
  res.map $\leftarrow$ m  '\ote{1}'
  res.sendasPermitidas $\leftarrow$   '\ote{1}' 
  itLista it $\leftarrow$ estaciones(m)   '\ote{1}'
  res.robotsEnEstacion $\leftarrow$ crearDicc()   '\ote{1}'
  while haySiguientes?(it) do '\ote{1}*\#(estaciones(m))'
   Definir(Siguiente(it), Vacia(), res.robotsEnEstacion) '\ote{|siguiente(it)|}'
  endwhile
  
'$\ofi{O( \#(estaciones(m)) * E_m )} donde E_m es el tamano de la estacion mas grande$'
\end{lstlisting}

  \begin{lstlisting}[mathescape]
  '\alg{iEntrar}{\In{ts}{dictt(string, bool)}, \In{e}{string}, \Inout{c}{ciudad}}{}'
    nat rurActual '$\leftarrow$' c.'$\#$'robots '\ote{1}'
    c.#robots '$\leftarrow$' c.'$\#$'robots + 1 '\ote{1}'
    Agregar(0, c.'$\#$'infracciones) '\ote{|c.$\#$infracciones|}'
    Agregar(e, c.estacion) '\ote{|c.estaciones|}'
    Agregar(ts, c.tags) '\ote{|c.tags|}'
    Agregar(true, c.vivos) '\ote{|c.vivos|}'
    dicc_T estacionesPermitidas '$\leftarrow$' nuevoDict() '\ote{1}'
    itLista(tupla(string, string)) iteradorSendas '$\leftarrow$' itSendas(c.mapa) '\ote{1}'
    while hayProximo(iteradorSendas): '\ote{S}'
      string estacion1 '$\leftarrow$' primero(siguiente(iteradorSendas)) '\ote{1}'
      string estacion2 '$\leftarrow$' segundo(siguiente(iteradorSendas)) '\ote{1}'
      if $\neg$ Definido?(estacionesPermitidas, estacion1): '\ote{|estacion1|}'
        Definir(estacion1, nuevoDict(), estacionesPermitidas) '\ote{|estacion1|}'
      endif  
      restr restriccion '$\leftarrow$' Restriccion(estacion1, estacion2, c.mapa) '\ote{|estacion1|+|estacion2|}'
      bool result '$\leftarrow$' Verifica?(restriccion, ts) '\ote{|restriccion|}'
      Definir(estacion2, result, Obtener(estacionesPermitidas, estacion1)) '\ote{|estacion1|+|estacion2|}'
      Avanzar(iteradorSendas) '\ote{1}'
    Agregar(c.estacionesPermitidas, estacionesPermitidas) '\ote{|c.estacionesPermitidas|}'
    colaPrior heapEstacion '$\leftarrow$' Obtener(e, c.robotsEnEstacion) '\ote{|e|}'
    itColaPrior '$\leftarrow$' Agregar((rurActual, 0), heapEstacion) '\ote{log(|heapEstacion|)}'
    Agregar(c.sacarRobotDeEstacion, iterador) '\ote{|c.sacarRobotDeEstacion|}'
  '\ofi{O(N_{total} + S * (|e_{m}| + |R|) + |e|)}'
  \end{lstlisting}

	Complejidad de iEntrar:
	Por el invRep de heap sabemos que
	|c.'$\#$'infracciones| = |c.estaciones| = |c.tags| = |c.vivos| = |c.estacionesPermitidas|  = |c.sacarRobotDeEstacion|= $N_{total}$
	Por lo tanto, llamar a Agregar para cada una de esas listas, agrega un $N_{total}$ a la complejidad final.
	Sabemos que itSendas devuelve un iterador de tupla(string, string), donde cada tupla representa cada pareja de estaciones que esta conectada, 
	por lo tanto, el while tiene $S$ ciclos. Sabemos que cada ciclo tiene complejidad O(|estacion1| + |estacion2| + |restriccion|).
	Entonces, si R es la restriccion de mayor tama�o en todo el mapa, y $e_{m}$ es la estacion de mayor longitud en el mapa, podemos decir que la
	complejidad de la parte del while, le agrega un O(S * (|$e_{m}$| + |R|)) a la complejidad total.
	La parte de objetener el heap dada la estacion, le agrega un total de |e| a la complejidad total.
	Tambien, insertar en la heap el nuevo elemento, toma O(log($N_{e}$)), donde $N_{e}$ es la cantidad de robots en el heap de la estacion e. 
	Como $N_{e}$ < $N_{total}$, podemos decir que le agrega un O(log($N_{e}$)) a la complejidad total.
	Resumiendo, la complejidad final es:
		O($N_{total}$ + S * (|$e_{m}$| + |R|) + |e| + log($N_{total}$)) = O($N_{total}$ + S * (|$e_{m}$| + |R|) + |e|)
	

  \begin{lstlisting}[mathescape]
  '\alg{iMover}{\In{u}{nat}, \In{e}{string}, \Inout{c}{ciudad}}{}'
    string estacionAnterior = c.estacion[u] '\ote{1}'
    string min '\ote{1}'
    string max '\ote{1}'
    if estacionAnterior < e '\ote{|estacionAnterior| + |e|}'
      min = estacionAnterior '\ote{1}'
      max = e '\ote{1}'
    else
      min = e '\ote{1}'
      max = estacionAnterior '\ote{1}'
    eliminar(c.sacarRobotDeEstacion[u]) '\ote{log($N_{e}$)}'
    bool result = Obtener(min, Obtener(max, c.estacionesPermitidas[u])) '\ote{|estacionAnterior| + |e|}'
    if not result '\ote{1}'
      c.#infracciones[u] = c.#infracciones + 1 '\ote{1}'
    colaPrior heapEstacion '$\leftarrow$' Obtener(e, c.robotsEnEstacion) '\ote{|e|}'
    itColaPrior '$\leftarrow$' Agregar((rurActual, c.#infracciones[u]), heapEstacion) '\ote{log(|heapEstacion|)}'
    c.sacarRobotDeEstacion[u] = iterador) '\ote{1}'
    c.estacion[u] = e  '\ote{1}'
  '\ofi{ O(log(N_e) + log(N_{estacionAnterior}) + $|$e$|$ + $|$estacionAnterior$|$)}'
  \end{lstlisting}


  \begin{lstlisting}[mathescape]
  '\alg{iInspeccion}{\In{e}{string}, \Inout{c}{ciudad}}{}'
    colaPrior heapEstacion = Obtener(e, c.robotsEnEstacion)  '\ote{|e|}'
    nat rur = Desencolar(heapEstacion) '\ote{log($N_{e}$)}'
    vivos[rur] = false '\ote{1}'
  '\ofi{O(log(N_e) + $|$e$|$)}'
  \end{lstlisting}


\begin{lstlisting}[mathescape]
'\alg{iRobots}{\In{c}{estr}}{itVectorNat}'

  res $\leftarrow$ CrearIt(c.vivos) '\ote{1}'
  
'\ofi{O(1)}'
\end{lstlisting}

\begin{lstlisting}[mathescape]
'\alg{iEstaciones}{\In{c}{estr}}{itVector(string)}'

    res $\leftarrow$ Estaciones($c$.mapa) '\ote{1}'
  
'\ofi{O(1)}'
\end{lstlisting}
%crear it de itlista%

\begin{lstlisting}[mathescape]
'\alg{iInfracciones}{\In{n}{nat}, \In{c}{estr}}{nat}'

    res $\leftarrow$ c.#infracciones[n] '\ote{1}'
  
'\ofi{O(1)}'
\end{lstlisting}

\begin{lstlisting}[mathescape]
'\alg{iEstacion}{\In{n}{nat}, \In{c}{estr}}{string}'

    res $\leftarrow$ c.estacion[n] '\ote{1}'
  
'\ofi{O(1)}'
\end{lstlisting}

\begin{lstlisting}[mathescape]
'\alg{iTags}{\In{n}{nat}, \In{c}{estr}}{$conj_T()$}'

    res $\leftarrow$ c.tags[n] '\ote{1}'
  
'\ofi{O(1)}'
\end{lstlisting}

\subsubsection{Algoritmos de itVectorNat}

\begin{lstlisting}[mathescape]
'\alg{iCrearIt}{\In{v}{vector(bool)}}{itVectorNat}'

  res.elementos $\leftarrow$ $v$ '\ote{1}' 
  res.actual $\leftarrow$ 0 '\ote{1}'
  
'\ofi{O(1)}'
\end{lstlisting}

\begin{lstlisting}[mathescape]
'\alg{iHayM\'as?}{\In{it}{itVectorNat}}{bool}'

  res $\leftarrow$ false '\ote{1}' 
  nat i = it.actual   '\ote{1}' 
  while (i < longitud(it.elementos) $\wedge$ $\neg$ res)   '\ote{longitud(it.elementos)}'
      if it.elementos[i] = true then  '\ote{1}'
          res = true  '\ote{1}'
      endif
      i++ '\ote{1}'
  
'\ofi{O(longitud(it.elementos))}'
\end{lstlisting}

\begin{lstlisting}[mathescape]
'\alg{iProximo}{\Inout{it}{itVectorNat}}{nat}'

    bool encontreProximo '\ote{1}'
    nat i = it.actual '\ote{1}'
    while (i < longitud(it.elementos) $\wedge$ $\neg$ encontreProximo)  '\ote{longitud(it.elementos)}'
        if it.elementos[i] = true then '\ote{1}'
            it.actual = i '\ote{1}'
            res $\leftarrow$ i
        endif
        i++ '\ote{1}'
  
'\ofi{O(longitud(it.elementos))}'
\end{lstlisting}


\end{document}
