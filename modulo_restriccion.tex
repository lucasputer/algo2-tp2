\section{M\'odulo Restriccion}
\subsection{Interfaz}

\textbf{se explica con}: Restriccion

  \textbf{usa}: tags, $dicc_t$(tags,bool), bool
  
  \textbf{genero}: genero
  
\subsubsection{Operaciones de Restriccion}

\InterfazFuncion{CrearRestriccion}{\In{t}{string}}{restr}
  [t != AND $\wedge$ t != OR $\wedge$ t != NOT $\wedge$ 0 < $|$t$|$ < 65]
  {$res \igobs$ $<t>$}
  [O($|$r$|$)]
  [Crea una nueva restriccion en base a un tag unico] \\

\InterfazFuncion{Negar}{\In{r}{restr}}{restr}
  [r $\igobs$ $r_0$]
  {r $\igobs$ NOT $r_0$}
  [O($|$r$|$)]
  [Niega la restriccion dada, generando otra] \\

\InterfazFuncion{And}{\In {r}{restr}, \In {rOtra}{restr}}{restr}
  [r $\igobs$ $r_0$]
  {r $\igobs$ $r_0$ AND rOtra}
  [O($|$r$|$)]
  [Hace and de las dos restricciones, generando otra.] \\ 

\InterfazFuncion{Or}{\In {r}{restr}, \In {rOtra}{restr}}{restr}
  [r $\igobs$ $r_0$]
  {r $\igobs$ $r_0$ Or rOtra}
  [O($|$r$|$)]
  [Hace or de las dos restricciones, generando otra.] \\

  \InterfazFuncion{Copiar}{\In {r}{restr}}{restr}
  [true]
  {res $\igobs$ r}
  [O($|$r$|$)]
  [Crea una copia de la restricci\'on] \\

\InterfazFuncion{Verifica}{\In {r}{restr},\In {ts}{$dicc_t$(tags,bool)}}{bool}
  [True]
  {res == verifica?(ts,r)}
  [O($|$r$|$)]

\subsection{Representaci\'on}
\subsubsection{Representaci\'on de Restricci\'on}

Representamos cada nodo del arbol con una clave y dos punteros a nodos hijos. Si la clave esta entre los strings (AND, OR, NOT), se considera al nodo como un nodo de operacion, y por lo tanto, si la clave es AND u OR sus dos punteros no pueden ser NULL, y si la clave es NOT, su puntero derecho debe ser NULL y su puntero izquiero no. Los nodos cuya clave no este entre las palabras (AND, OR y NOT) deberan tener como NULL a sus dos punteros.

\begin{Estructura}{restr}[restr\_tup]
    \begin{Tupla}[restr\_tup]
      \tupItem{nombre}{string}
      \tupItem{izq}{puntero(restr)}
      \tupItem{der}{puntero(restr)}
      \tupItem{tam}{nat}
    \end{Tupla}
    \end{Estructura} 
    
\subsubsection{Invariante de Representaci\'on}

\renewcommand{\labelenumi}{(\Roman{enumi})}
\begin{enumerate}
\item El tama\~no de un nodo se define como la suma del tama\~no de sus hijos + 1.
\item El arbol no debe tener ciclos, o sea, existe n en Nat tal que n > al tama\~no de cada uno de los nodos del \'arbol
\item Los nodos hoja tienen clave disinta de (AND, OR, NOT)
\item Los nodos NO hoja tienen su clave entre los valores (AND, OR, NOT)
\item Si el nodo tiene clave entre los valores (AND, OR) sus dos punteros son distintos de NULL
\item Si el nodo tiene como clave NOT, su puntero derecho debera ser NULL y su puntero izquierdo debera ser distinto de NULL
\item Cada nodo es un arbol en si mismo
\end{enumerate} 
    
\subsubsection{Funci\'on de Abstracci\'on}
Abs : restr\_tup  $\rightarrow$ Restriccion \\

Abs(e) $\igobs$ r: restr\_tup \  |\  $\forall$ cj : conj(tag) \\
\phantom{pala}Verifica?(r, cj) $\Leftrightarrow$ Verifica$?$(e, cj)


\subsection{Algoritmos}

\lstset{style=alg}

\begin{lstlisting}[mathescape]
'\alg{iCrearRestriccion}{\In{t}{string}}{restr}'
    
    restr restriccion
    restriccion.nombre $\leftarrow$ t '\ote{1}'
    restriccion.izq $\leftarrow$ NULL '\ote{1}'
    restriccion.der $\leftarrow$ NULL '\ote{1}'
    restriccion.tam $\leftarrow$ 1 '\ote{1}'
    res $\leftarrow$ *restriccion  '\ote{1}'
    
'\ofi{O(1)}'
\end{lstlisting}

\begin{lstlisting}[mathescape]
'\alg{iNegar}{\Inout{r}{restr}}{restr}'
    
    restr not $\rightarrow$  <"NOT",NULL,NULL,NULL> '\ote{1}'
    not$\rightarrow$izq  $\leftarrow$ iCopiar(r) '\ote{r.tam}'
    not.tam  $\leftarrow$ r.tam + 1 '\ote{1}'
    ret $\leftarrow$ not '\ote{1}'
   
'\ofi{O(r.tam)}'
\end{lstlisting}

\begin{lstlisting}[mathescape]
'\alg{iOr}{\In{r}{restr}, \In{rOtra}{restr}}{restr}'

    restr or $\rightarrow$  <"OR",NULL,NULL,NULL> '\ote{1}'
    or$\rightarrow$izq  $\leftarrow$ iCopiar(r) '\ote{r.tam}'
    or$\rightarrow$ der  $\leftarrow$ iCopiar(rOtra) '\ote{r.tam}'
    or.tam  $\leftarrow$ r.tam + rOtra.tam + 1 '\ote{1}'
    ret $\leftarrow$ or '\ote{1}'
    
'\ofi{O(r.tam)}'
\end{lstlisting}

\begin{lstlisting}[mathescape]
'\alg{iAnd}{\Inout {r}{restr}, \In {rOtra}{restr}}{restr}'
    
    estr and $\rightarrow$  <"AND",NULL,NULL,NULL> '\ote{1}'
    and$\rightarrow$izq  $\leftarrow$ iCopiar(r) '\ote{r.tam}'
    and$\rightarrow$ der  $\leftarrow$ iCopiar(rOtra) '\ote{r.tam}'
    and.tam  $\leftarrow$ r.tam + rOtra.tam + 1 '\ote{1}'
    ret $\leftarrow$ and '\ote{1}'
    
'\ofi{O(r.tam)}'
\end{lstlisting}

\begin{lstlisting}[mathescape]
'\alg{iVerifica}{\In {r}{restr},\In {ts}{$dicc_t$(tags,bool)}}{bool}'
    
   if r.nombre = "NOT" : '\ote{1}'
     res $\leftarrow$ ($\neg$ iVerifica(r.izq,ts)) '\ote{r.tam}'
   else if r.nombre  = "AND": '\ote{1}'
     res $\leftarrow$ (iVerifica(r.izq,ts) && iVerifica(r.der,ts)) '\ote{r.tam}'
   else if r.nombre = "OR": '\ote{1}'
     res $\leftarrow$ (iVerifica(r.izq,ts) || iVerifica(r.der,ts)) '\ote{r.tam}'
   else:
    res $\leftarrow$ definido?(r.nombre, ts)
   endif
       
'\ofi{O(r.tam)}'
\end{lstlisting}

\subsubsection{Complejidad de iVerifica}
Calcularemos la complejidad en base a m, la cantidad de nodos del \'arbol. 
En el peor caso de iVerifica (AND o OR) nos quedamos con dos subproblemas que tienen la mitad de tama\~no que el original, as\'i que
a = 2 y c = 2.
M\'as all\'a de esto, se hace una cantidad constante de comparaciones y asignaciones, por lo que podemos decir que f(n) $\in$ O(1).
Usamos el primer caso del Teorema Maestro, porque $n^{log_2(2)-\epsilon}$ = n$^{1-\epsilon}$. Entonces, si elegimos, por ejemplo
$\epsilon$ = 1, nos queda O($n^0$) = O(1) y sabemos que f(n) tiene esa complejidad.
Por lo tanto, utilizando el primer caso tenemos que T(n) $\in$ O($m^{log_2(2)}$) = O(m).

\begin{lstlisting}[mathescape]
'\alg{iCopiar}{\In {r}{restr}}{restr}'
    
   restr copia
   copia.nombre  $\leftarrow$ r.nombre '\ote{64}'
   copia.tam  $\leftarrow$ r.tam '\ote{1}'
   copia.izq  $\leftarrow$ NULL '\ote{1}'
   copia.der  $\leftarrow$ NULL '\ote{1}'
   if r.izq != NULL
    copia.izq  $\leftarrow$  iCopiar(&(r.izq)) '\ote{r.izq$\rightarrow$tam}'
   end if
   if r.der != NULL
    copia.der  $\leftarrow$  iCopiar(&(r.der)) '\ote{copy(r.der$\rightarrow$tam)}'
   end if 
       
'\ofi{O(r.tam)}'
\end{lstlisting}

\subsubsection{Complejidad de iCopiar}
La demostraci\'on de la complejidad es an\'aloga a la de iVerifica ya que es la misma recursi\'on y el caso base es O(1).




